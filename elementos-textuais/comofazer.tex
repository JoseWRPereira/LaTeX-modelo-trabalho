\chapter{Como fazer?}
\label{cap:como_fazer}



\section{Motivação}
\label{sec:motivacao}


\section{Objetivos}
\label{sec:objetivos}



\subsection{Objetivo Geral}
\label{sec:objetivo-geral}


\subsection{Objetivos Específicos}
\label{sec:objetivos-especificos}



	\begin{alineas}
		\item Primeira
		\item Segunda
		\item Terceira
	\end{alineas}


\section{Figura}
\label{sec:figura}
	
	\begin{figure}[h!]
		\centering
		\Caption{\label{fig:exemplo-1} Lorem ipsum dolor sit amet, consectetur adipiscing elit. Suspendisse commodo lectus et augue elementum varius.}	
		\UECEfig{}{
			\fbox{\includegraphics[width=8cm]{figuras/figura-1}}
		}{
			\Fonte{Elaborado pelo autor}
		}	
	\end{figure}


\section{Tabela}
\label{sec:tabela}

	\begin{table}[h!]	
	\centering
	\Caption{\label{tab:exemplo-1} Duis faucibus, enim quis tincidunt pellentesque, nisl leo varius nulla, vitae tempus dui mauris ac ante purus lorem}		
	\UECEtab{}{
		\begin{tabular}{cll}
			\toprule
			Ranking & Exon Coverage & Splice Site Support \\
			\midrule \midrule
			E1 & Complete coverage by a single transcript & Both splice sites\\
			E2 & Complete coverage by more than a single transcript & Both splice sites\\
			E3 & Partial coverage & Both splice sites\\
			E4 & Partial coverage & One splice site\\
			E5 & Complete or partial coverage & No splice sites\\
			E6 & No coverage & No splice sites\\
			\bottomrule
		\end{tabular}
	}{
		\Fonte{Elaborado pelo autor}
	}
	\end{table}


\section{alineascompont}
\label{sec:alineascompont}


\acrlong{DATASUS},

\acrlong{DNV},

\acrlong{DO},


\begin{alineascomponto}
	\item Primeira
	\item Segunda
	\item Terceira
	\begin{subalineascomponto}
		\item Primeira
		\item Segunda
	\end{subalineascomponto}
\end{alineascomponto}




O autor \cite{lamport1986latex} e \cite{Maia2011} e \cite{ibge23} \lipsum[2] 

\begin{table}[h!]
	\Caption{\label{tabela-ibge} Um Exemplo de tabela alinhada que pode ser longa ou curta, conforme padrão IBGE. conforme padrão IBGE. conforme padrão IBGE. conforme padrão IBGE. conforme padrão IBGE. conforme padrão IBGE. conforme padrão IBGE. conforme padrão IBGE. conforme padrão IBGE. conforme padrão IBGE. conforme padrão IBGE.}%
	\IBGEtab{}{%
		\begin{tabular}{ccc}
			\toprule
			Nome & Nascimento & Documento \\
			\midrule \midrule
			Maria da Silva & 11/11/1111 & 111.111.111-11 \\
			Maria da Silva & 11/11/1111 & 111.111.111-11 \\
			Maria da Silva & 11/11/1111 & 111.111.111-11 \\
			\bottomrule
		\end{tabular}%
	}{%
		\Fonte{Produzido pelos autores}%
		\Nota{Esta éuma nota, que diz que os dados são baseados na
			regressão linear.}%
		\Nota[Anotações]{Uma anotação adicional, seguida de várias outras.}%
	}
\end{table}

\cite{Huetal2000} 

\section{Exemplo de Algoritmos e Figuras}
\label{sec:exemplo-de-algoritmos-e-figuras}



\begin{algorithm}[h!]
	\SetSpacedAlgorithm
	\caption{\label{exemplo-de-algoritmo}Como escrever algoritmos no \LaTeX2e}
	\Entrada{o proprio texto}
	\Saida{como escrever algoritmos com \LaTeX2e }
	\Inicio{
		inicializa\c{c}\~ao\;
		\Repita{fim do texto}{
			leia o atual\;
			\Se{entendeu}{
				vá para o próximo\;
				próximo se torna o atual\;}
			\Senao{volte ao início da seção\;}
		}
	}	
\end{algorithm}







Exemplo de alíneas com números:

\begin{alineascomnumero}
	\item Primeira
	\item Segunda
\end{alineascomnumero}


\begin{table}[h!]	
	\centering
	\Caption{\label{tab:internal}Internal exon scores}	
	\IBGEtab{}{
		\begin{tabular}{cll}
			\toprule
			Ranking & Exon Coverage & Splice Site Support\\
			\midrule \midrule
			E1 & Complete coverage by a single transcript & Both splice sites\\
			E2 & Complete coverage by more than a single transcript & Both splice sites\\
			E3 & Partial coverage & Both splice sites\\
			E4 & Partial coverage & One splice site\\
			E5 & Complete or partial coverage & No splice sites\\
			E6 & No coverage & No splice sites\\
			\bottomrule
		\end{tabular}
	}{
		\Fonte{os autores}
	}
\end{table}

Referenciando a \autoref{tab:internal} 

\index{figuras}Figuras podem ser criadas diretamente em LaTeX, como o exemplo da \ref{fig-grafico-1}.

\begin{figure}[h!]
	\centering
	\Caption{\label{fig-grafico-1}Produção anual das dissertações de mestrado e teses de doutorado entre os anos de 1990 e 2008}		
	\IBGEtab{}{
		\fbox{\includegraphics[scale=0.5]{figuras/figura-3}}
	}{
		\Fonte{os autores}
	}
\end{figure}

Ou então figuras podem ser incorporadas de arquivos externos, como é o caso da \autoref{fig-grafico-1}.  para manter a coerência no uso de software livre (já que você está usando LaTeX e abnTeX),  teste a ferramenta InkScape\index{InkScape}. ao CorelDraw\index{CorelDraw} ou ao Adobe Illustrator\index{Adobe! Illustrator}, \index{Gimp}Gimp. Ele é uma alternativa livre ao Adobe Photoshop\index{Adobe! Photoshop}.

\section{Usando Fórmulas Matemáticas}



\begin{equation}
	\begin{aligned}
		x = a_0 + \cfrac{1}{a_1
			+ \cfrac{1}{a_2
				+ \cfrac{1}{a_3 + \cfrac{1}{a_4} } } }
	\end{aligned}
\end{equation}


\begin{equation}
	\begin{aligned}
		k_{n+1} = n^2 + k_n^2 - k_{n-1}
	\end{aligned}
\end{equation}


\begin{equation}
	\begin{aligned}
		\cos (2\theta) = \cos^2 \theta - \sin^2 \theta
	\end{aligned}
\end{equation}


\begin{equation}
	\begin{aligned}
		A_{m,n} =
		\begin{pmatrix}
			a_{1,1} & a_{1,2} & \cdots & a_{1,n} \\
			a_{2,1} & a_{2,2} & \cdots & a_{2,n} \\
			\vdots  & \vdots  & \ddots & \vdots  \\
			a_{m,1} & a_{m,2} & \cdots & a_{m,n}
		\end{pmatrix}
	\end{aligned}
\end{equation}


\begin{equation}
	\begin{aligned}
		f(n) = \left\{ 
		\begin{array}{l l}
			n/2 & \quad \text{if $n$ is even}\\
			-(n+1)/2 & \quad \text{if $n$ is odd}
		\end{array} \right.
	\end{aligned}
\end{equation}






\section{Usando Algoritmos}


\begin{algorithm}[h!]
	\SetSpacedAlgorithm
	\caption{\label{alg:algoritmo_de_colonica_de_formigas}Algoritmo de Otimização por Colônia de Formiga}
	\Entrada{Entrada do Algoritmo}
	\Saida{Saida do Algoritmo}
	\Inicio{
		Atribua os valores dos parâmetros\;
		Inicialize as trilhas de feromônios\;
		\Enqto{não atingir o critério de parada}{
			\Para{cada formiga}{
				Construa as Soluções\;
			}
			Aplique Busca Local (Opcional)\;
			Atualize o Feromônio\;
		}	
	}		
\end{algorithm}


\section{Usando Código-fonte}


\lstinputlisting[language=C++,caption={Hello World em C++}]{figuras/main.cpp}


\begin{lstlisting}[language=Java,caption={Hello World em Java}]
	public class HelloWorld {
		public static void main(String[] args) {
			System.out.println("Hello World!");
		}
	}
\end{lstlisting}


\section{Usando Teoremas, Proposições, etc}


\begin{teo}[Pitágoras]
	Em todo triângulo retângulo o quadrado do comprimento da
	hipotenusa é igual a soma dos quadrados dos comprimentos dos catetos.
\end{teo}



\begin{teo}[Fermat]
	Não existem inteiros $n > 2$, e $x, y, z$ tais que $x^n + y^n = z$
\end{teo}


\begin{prop}
	Para demonstrar o Teorema de Pitágoras...
\end{prop}


\begin{exem}
	Este é um exemplo do uso do ambiente exe definido acima.
\end{exem}


\begin{xdefinicao}
	Definimos o produto de ...
\end{xdefinicao}


\section{Usando Questões}



\begin{questao}
	\item Esta é a primeira questão com alguns itens:
	\begin{enumerate}
		\item Este é o primeiro item
		\item Segundo item
	\end{enumerate}
	\item Esta é a segunda questão:
	\begin{enumerate}
		\item Este é o primeiro item
		\item Segundo item
	\end{enumerate}
	\item Lorem ipsum dolor sit amet, consectetur adipiscing elit. Nunc dictum sed tortor nec viverra. consectetur adipiscing elit. Nunc dictum sed tortor nec viverra.
	\begin{enumerate}
		\item consectetur
		\item adipiscing
		\item Nunc
		\item dictum
	\end{enumerate}
\end{questao}

\section{Citações}

\subsection{Documentos com três autores}

Quando houver três autores na citação, apresentam se os três, separados por ponto e vírgula, caso estes estejam após o texto. Se os autores estiverem incluídos no texto, devem ser separados por vírgula e pela conjunção "e".

\citeautoronline{tresautores}

\cite{tresautores}

\subsection{Documentos com mais de três autores}
Havendo mais de três autores, indica-se o primeiro seguido da expressão \textit{et al.} (do latim \textit{et alli}, que significa e outros), do ano e da página.

\citeautoronline{quatroautores}

\cite{quatroautores}

\subsection{Documentos de vários autores}

Havendo    citações    indiretas de    diversos    documentos    de    vários    autores, mencionados  simultaneamente e  que  expressam  a  mesma  ideia,  separam-se  os  autores  por ponto e vírgula, em ordem alfabética.

\cite{tresautores, quatroautores}

\section{Notas de Rodap\'{e}}

Deve-se utilizar o sistema autor-data para as  citações no texto e o numérico para notas explicativas\footnote{Veja - se como exemplo desse tipo de abordagem o estudo de Netzer (1976)}. As notas de rodapé podem e devem ser alinhadas, a partir da segunda linha da mesma nota, abaixo da primeira letra da primeira palavra, de forma a destacar o expoente \footnote{Encontramos  esse  tipo  de  perspectiva  na  2ª  parte  do  verbete  referido  na  nota  anterior,  em  grande  parte  do estudo de Rahner (1962).} e sem espaço entre elas e com fonte menor (tamanho 10).
